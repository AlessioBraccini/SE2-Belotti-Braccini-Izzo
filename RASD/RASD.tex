\documentclass[table, 12pt]{article}
\usepackage[T1]{fontenc}
\usepackage[utf8]{inputenc}
\usepackage[english]{babel}
\usepackage{graphicx}
\usepackage{titlesec}
\usepackage{hyperref}
\usepackage[table]{xcolor}

\hyphenation{Te-lan-ga-na}
\hyphenation{an-a-lys-ing}
\titleformat{\paragraph}
{\normalfont\normalsize\bfseries}{\theparagraph}{1em}{}
\titlespacing*{\paragraph}
{0pt}{3.25ex plus 1ex minus .2ex}{1.5ex plus .2ex}

\begin{document}
\begin{titlepage}
    \centering
    {\scshape\large AY 2021/2022 \par}
    \vfill
    \includegraphics[width=100pt]{assets/logo-polimi-new}\par\vspace{1cm}
    {\scshape\LARGE Politecnico di Milano \par}
    \vspace{1.5cm}
    {\huge\bfseries RASD\@: Requirement Analysis
        and Specification Document \par}
    \vspace{2cm}
    {\Large {Ottavia Belotti\quad Alessio Braccini\quad Riccardo Izzo}\par}
    \vfill
    {\large Professor\par
        Elisabetta \textsc{Di Nitto}}
    \vfill
    {\large \textbf{Version 1.0}\\ \today \par}
\end{titlepage}

\hypersetup{%
    pdfborder = {0 0 0}
}

\thispagestyle{plain}
\pagenumbering{gobble}
\mbox{}
\newpage
\pagenumbering{roman}
\tableofcontents
\newpage
\pagenumbering{arabic}

\section{Introduction}
\emph{Data-dRiven PrEdictive FArMing}, also known as \emph{DREAM}, is a project presented by UNDP India and Healthsites initiative, promoted by Telangana's government.
The aim of the project is to enhance the farm system and the entire food supply chain with an IT supporting application. 
This arises from modern challenges like climate change and the foreseen population growth that have underlined the critical issues of the modern system making necessary a complete overhaul.


\subsection{Purpose} %goals of the project
\emph{DREAM} aims to support work categories involved into the farming industry by providing them relevant and up-to-date data about the farm activity's performance. 
The main stakeholders are: Telangana's policy makers, farmers and agronomists.
The goal is to develop a data-driven application with the help of IT partners.
Telangana's state already collect important data concerning wheather forecast, these data are publicly available with a live rainfall map on the official government website.
Other data can be collected through humidity sensors deployed all over the territory and through the water irrigation system.

Agriculture has a main role in India's economy, more than half of the population depends on it and about a fifth is below the poverty line.
Furthermore, as a significant increment in population is expected for 2050 (\emph{UN}'s esteem), food demand is going to significantly increase.
Telangana needs an efficient application to increase the general productivity of the farm system.

The user base is expected to be the entire population of Telangana, starting with those who works in the agricolture sector up to normal citizens.

\subsection{Scope} %analysis of the world and shared phenomena
%ADD: basic service and advanced functionalities
\subsubsection*{Phenomena controlled by the Machine}
\rowcolors{2}{gray!50}{}
\begin{tabular}{|c|c|c|}
    \hline
    \textbf{ID} & \textbf{Phenomenom} & \textbf{Shared} \\\hline\hline
    M1 & Check username and password & No \\\hline
    M2 & Analysis of best practices & No\\\hline
    M3 & Analysis of weather data & No \\\hline
    M4 & Visualize data concerning weather, land, performance & Yes\\\hline
    \hline
\end{tabular}

\subsubsection*{Phenomena controlled by the World}
\begin{center}
    \rowcolors{2}{gray!50}{}
    \begin{tabular}{|c|c|c|}
        \hline
        \textbf{ID} & \textbf{Phenomenom} & \textbf{Shared} \\\hline\hline
        W1 & User login & Yes\\\hline
        W2 & User share best practice & Yes\\\hline
        W3 & User ask for help on forum & Yes\\\hline
        W4 & Collect land data from sensor & Yes\\\hline
        W5 & User create topic in forum & Yes\\\hline 
        W6 & User insert post & Yes\\\hline
        W7 & User reply to a post & Yes \\\hline
        W8 & User update daily plan & Yes\\\hline
        W9 & User check weather forcast & Yes\\\hline
        \hline
    \end{tabular}
\end{center}

\subsubsection{Goals}
\underline{Telangana's policy makers}
\begin{enumerate}
    \item \textbf{Identification of well-performing farmers}\\
    Main goal of the policy makers is to identify farmers that are resilient to meteorological adverse events.
    This can be done comparing the productivity ratio defined as the produced amount per product in adverse condition over the amount in standard conditions.
    This farmers will receive special incentives and will be asked to help other farmers 
    with useful practices.
    \item \textbf{Identification of bad-performing farmers}\\
    Identify farmers that are performing bad using the productivity ratio, they are the ones that need to be helped 
    by the well-performing farmers.
    \item \textbf{Visualize the results of steering initiatives}\\
    Visualize and evaluate the results produced by the steering initiatives from agronomists and good farmers.
\end{enumerate}

\underline{Farmers}
\begin{enumerate}
    \item \textbf{Visualize data}\\
    Visualize important data like weather forecast and personalized suggestion about specific crops or fertilizers.
    All data are based on location and type of production.
    \item \textbf{Insert data}\\
    Insert data about their production, report every type of problems.
    \item \textbf{Request for help/suggestion}\\
    Farmers can request help with a text message that will be sent directly to the agronomists responsible of the area.
    \item \textbf{Create discussion forums}\\
    Create forums to discuss with the other farmers.
    In this section the creator can choose the name of the forum and invite all the desirable partecipants.
\end{enumerate}

\underline{Agronomists}
\begin{enumerate}
    \item \textbf{Insert area}\\
    Insert the area of responsibility for the agronomist.
    \item \textbf{Receive request for help/suggestion}\\
    Here the agronomist can manage all the incoming request for help or suggestion.
    This can be done with a specific section where the agronomist can visualize the message and answer it.
    \item \textbf{Visualize area stats}\\
    Visualize data about whether forecast or a list of best-performing farmers.
    The list of best-performing farmers is based on the productivity over a selected period of time.
    \item \textbf{Visualize and update daily plan}\\
    The daily plan consists in a list of farms to be visited during the day.
    Every farm must be visited at least twice a year with particular attention to the under-performing ones that 
    should be visited more often.
    \item \textbf{Confirm the daily plan}\\
    Confirm the daily plan at the end of the day or update it in case of deviations.
\end{enumerate}

\subsection{Definitions, acronyms, abbreviations}
\subsubsection*{Definitions}
\subsubsection*{Acronyms}
\begin{itemize}
    \item \textbf{RASD}: Requirement Analysis and Specification Document
    \item \textbf{DREAM}: \emph{Data-driven predictive farming} project
    \item \textbf{Telangana}: Indian state promoting the \emph{DREAM} project
\end{itemize}
\subsection{Revision history}
\subsection{Reference documents}
\begin{itemize}
    \item Specification document: "Assignment RDD AY 2021-2022"
    \item Alloy documentation: https://alloytools.org/documentation.html
    \item UML documentation: https://www.uml-diagrams.org/
    \item BPMN documentation: https://www.bpmn.org/
    \item Paper: "The World and the Machine" by M. Jackson and P. Zave
\end{itemize}

\subsection{Document structure}
\begin{itemize}
    \item \textbf{Section 1} gives an introduction about the problem to tackle and about which functionalities will be implemented in the final product in order to solve it.
    \item \textbf{Section 2} contains the overall description of the whole project, presenting it in a more formal way through class diagrams which will contain the backbone blocks that will build the final application. Furthermore, there will be presented the so-called \emph{actors} who are the ones that will use the application, the expected functionalities and the domain assumptions taken in consideration throughout the whole project, from the specification phase to the actual developing phase.
    \item \textbf{Section 3} delves deeply into the technical aspects of the topics presented in \emph{Section 2}, in order to be more useful for the development, by providing a standard interfaces' system \textit{a priori} to  stick to during the project implementation. It will show functional and non-functional requirements. The former will be presented through some use-cases and scenarios as meaningful examples; while the latter will be disclosed by analysing performance, design and software system features that the project will have.
    \item \textbf{Section 4} presents the Alloy code briefly explainig the purpose of it in modeling the given problem.
\end{itemize}

\section{Overall Description}
\subsection{Product perspective}
\subsection{Product functions}
\subsubsection{Sign-up and shared functions}
\begin{itemize}
    \item \textbf{Sign-up:} let the user sign-up thorugh an email and a password, creating a profile tailored for the user's job. Specify the area in which they live and what type of cultivation they manage.
\end{itemize}
\subsubsection{Policy makers functions}
\begin{itemize}                                 
    \item \textbf{Visualize relevant data and initiative: }let the policy makers know a variety of different data like the performances of the farmers by grouping them in a rank to know who are the farmers that are performing well and who are the worst one based on informations insert by them.
    Policy makers can also visualize the steering iniziative presented by the agronomist in a specific subsection of their view. (ok so che non è view ma non mi viene il nome)
\end{itemize}
\subsubsection{Farmers functions}
\begin{itemize}
    \item \textbf{Profile edit: }allow the farmers change their profile in order to upgrade information like: 
    \subitem Area: allow to change the area where farmers have their plantation;
    \subitem Plant type: allow to change the type of plant;
    \subitem Username and password: allow to change their username and password.

    \item \textbf{Manipulation of informations: }allow farmers to visualize every kind of their interest like the wheather forecast for the day or for the next few days or some suggestions about own crops and specific fertilizers.
    \item \textbf{Send message: }allow farmers to send messages to the agronomist.
    \item \textbf{Usage of the forum: }allow farmers to create a new topic or reply to a message in the dedicated forum.
\end{itemize}
\subsubsection{Agronomists functions}
\begin{itemize}
    \item \textbf{Area functions: }allow the agronomist to insert their responsability area and visualize the correspondant data like wheather forecast or the rank of the farmers.
    \item \textbf{Manage farmers requests: }farmers can send help or suggestion request to whom agronomist have to reply.
    \item \textbf{Manage daily plan: }allow the agronomist to make their daily plan by registring the incoming visits to the farmers. Some days before visits they can confirm the plan and send a notification to farmers or deviate to the plan and inform farmers of this change.   
\end{itemize}

\subsection{User characteristics}
The application has been thought for the three different user categories that follows:
\begin{itemize}
    \item \textbf{Policy makers} are government's employees that are in charge of analysing the general agricolture trends among all the districts in Telangana, then promote based state-wide policies to better the whole food system. Their main goal is not only to secure the current provision, but also to identify now the best practices that will lead to a flourishing food production in the future. By doing so, the plan is to grow more resiliant and profitable crops and prepare the lands to face future menaces, for instance the climate that is getting more hostile or the foreseen increment of the food demand. As a consequence, they want to be notified about the best performing farmers in order to contact them and get more insight from them about their procedures, with the aim to acquire best practises to be shared and applied on a larger scale. At the same time, they need to know who, on the other hand, is performing particularly badly, so that they can be given the help they need to better their results, since obtaining the foresaid goal requires the structure to run smoothly in all its parts. 
    Policy makers also need a feedback system that let them be aware of the true impact \textit{a posteriori} of the initiatives carried out by the agronomists in collaboration with the knowledge and practice of the best farmers.
    \item \textbf{Farmers} are interested in functionalities that will help with their day-to-day life at work, so they would like to receive in one place all the information about the weather to plan before hand the work day and useful data, like suggestions and news about the specific crop they cultivate, if some crop's illness is spreading in their area and how to treat it, which fertilizers boost the plant's production, etc. Moreover, they should insert data about their own production and ask for help to a regional agronomist through the app if it's needed. Being part of a larger community of people that share the same purpose (such as being more productive) brings more knowledge in general, so it's easier for the farmers to get in touch with their collegues that grow the same crops and might have faced the same challenges they do through the in-app forum. The feature allows them to enlarge their pool of acquaintances and brings them together online, even though they might be kilometers away from each other.
    \item \textbf{Agronomists} are the experts in the agricolture field, so the main function needed for them is the possibility of helping out the farmers that reach out to them. Each agronomist is in charge of a specific geographical area in Telangana, in order to be efficiently present on the territory in a fair and useful way according to the actual helping demand. In fact, they visit each farm spread among their area at least twice a year. That said, agronomists would like some functionalities that help them planning out their trips on the field in an simple yet flexible way. Furthermore, they would like to be notified of the farms' performace, especially the ones scoring poor results in order to plan their visits more often for those, depending on the problem their facing. Nonetheless, in order to make a complete and axhaustive report about the area productivity for the central government, they are also interested into acknowledging the top performing farms. %wheather forcast? go deeper in explaining daily plan funct?
\end{itemize}
\subsection{Assumptions, dependencies and constraints}

\section{Specific Requirements}
\subsection{External Interface Requirements}
\subsubsection{User Interfaces}
\subsubsection{Hardware Interfaces}
\subsubsection{Software Interfaces}
\subsubsection{Communication Interfaces}

\subsection{Functional Requirements}
\subsection{Performance Requirements}
Test.

\subsection{Design Constraints}
Test.

\subsubsection{Standards compliance}
\subsubsection{Hardware limitations}
\subsubsection{Any other constraint}

\subsection{Software System Attributes}
\subsubsection{Reliability}
Test.

\subsubsection{Availability}
Test.

\subsubsection{Security}
\subsubsection{Maintainability}
\subsubsection{Portability}

\section{Formal Analysis using Alloy}
\section{Effort Spent}
    \begin{tabular}{| c || c | c| c| c |}
        \hline
        Student & Time for S.1 & S.2 & S.3 & S.4 \\ \hline
        Ottavia Belotti & 1h & 2h & ? & ? \\
        Alessio Braccini & 2h & ? & ? & ? \\
        Riccardo Izzo & 2h & ? & ? & ? \\
        \hline
    \end{tabular}

\section{References}
\end{document}