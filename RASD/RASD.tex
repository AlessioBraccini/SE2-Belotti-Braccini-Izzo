\documentclass[table, 12pt]{article}
\usepackage[T1]{fontenc}
\usepackage[utf8]{inputenc}
\usepackage[english]{babel}
\usepackage{graphicx}
\usepackage{titlesec}
\usepackage{hyperref}
\usepackage[table]{xcolor}

\titleformat{\paragraph}
{\normalfont\normalsize\bfseries}{\theparagraph}{1em}{}
\titlespacing*{\paragraph}
{0pt}{3.25ex plus 1ex minus .2ex}{1.5ex plus .2ex}

\begin{document}
\begin{titlepage}
    \centering
    {\scshape\large AY 2021/2022 \par}
    \vfill
    \includegraphics[width=100pt]{assets/logo-polimi-new}\par\vspace{1cm}
    {\scshape\LARGE Politecnico di Milano \par}
    \vspace{1.5cm}
    {\huge\bfseries RASD\@: Requirement Analysis
        and Specification Document \par}
    \vspace{2cm}
    {\Large {Ottavia Belotti\quad Alessio Braccini\quad Riccardo Izzo}\par}
    \vfill
    {\large Professor\par
        Elisabetta \textsc{Di Nitto}}
    \vfill
    {\large \textbf{Version 1.0}\\ \today \par}
\end{titlepage}

\hypersetup{%
    pdfborder = {0 0 0}
}

\thispagestyle{plain}
\pagenumbering{gobble}
\mbox{}
\newpage
\pagenumbering{roman}
\tableofcontents
\newpage
\pagenumbering{arabic}

\section{Introduction}
\emph{Data-dRiven PrEdictive FArMing}, also known as \emph{DREAM}, is a project presented by UNDP India and Healthsites initiative, promoted by Telangana's government.
The aim of the project is to enhance the farm system and the entire food supply chain with an IT supporting application. 
This arises from modern challenges like climate change and the foreseen population growth that have underlined the critical issues of the modern system making necessary a complete overhaul.


\subsection{Purpose} %goals of the project
\emph{DREAM} aims to support work categories involved into the farming industry by providing them relevant and up-to-date data about the farm activity's performance. 
The main stakeholders are: Telangana's policy makers, farmers and agronomists.
The goal is to develop a data-driven application with the help of IT partners.
Telangana's state already collect important data concerning wheather forecast, these data are publicly available with a live rainfall map on the official government website.
Other data can be collected through humidity sensors deployed all over the territory and through the water irrigation system.

Agriculture has a main role in India's economy, more than half of the population depends on it and about a fifth is below the poverty line.
Furthermore, as a significant increment in population is expected for 2050 (\emph{UN}'s esteem), food demand is going to significantly increase.
Telangana needs an efficient application to increase the general productivity of the farm system.

The user base is expected to be the entire population of Telangana, starting with those who works in the agricolture sector up to normal citizens.

\subsection{Scope} %analysis of the world and shared phenomena

\subsubsection*{Phenomena controlled by the Machine}
\rowcolors{2}{gray!50}{}
\begin{tabular}{|c|c|c|}
    \hline
    \textbf{ID} & \textbf{Phenomenom} & \textbf{Shared} \\\hline\hline
    M1 & Check username and password & No \\\hline
    M2 & Analysis of best practices & No\\\hline
    M3 & Analysis of weather data & No \\\hline
    M4 & Visualize data concerning weather, land, performance & Yes\\\hline
    \hline
\end{tabular}

\subsubsection*{Phenomena controlled by the World}
\rowcolors{2}{gray!50}{}
\begin{tabular}{|c|c|c|}
    \hline
    \textbf{ID} & \textbf{Phenomenom} & \textbf{Shared} \\\hline\hline
    W1 & User login & Yes\\\hline
    W2 & User share best practice & Yes\\\hline
    W3 & User ask for help on forum & Yes\\\hline
    W4 & Collect land data from sensor & Yes\\\hline
    W5 & User create topic in forum & Yes\\\hline 
    W6 & User insert post & Yes\\\hline
    W7 & User reply to a post & Yes \\\hline
    W8 & User update daily plan & Yes\\\hline
    W9 & User check weather forcast & Yes\\\hline
    \hline
\end{tabular}

\subsubsection{Goals}
\underline{Telangana's policy makers}
\begin{enumerate}
    \item \textbf{Identification of well-performing farmers}\\
    Main goal of the policy makers is to identify farmers that are resilient to meteorological adverse events.
    This can be done comparing the productivity ratio defined as the produced amount per product in adverse condition over the amount in standard conditions.
    This farmers will receive special incentives and will be asked to help other farmers 
    with useful practices.
    \item \textbf{Identification of bad-performing farmers}\\
    Identify farmers that are performing bad using the productivity ratio, they are the ones that need to be helped 
    by the well-performing farmers.
    \item \textbf{Visualize the results of steering initiatives}\\
    Visualize and evaluate the results produced by the steering initiatives from agronomists and good farmers.
\end{enumerate}

\underline{Farmers}
\begin{enumerate}
    \item \textbf{Visualize data}\\
    Visualize important data like weather forecast and personalized suggestion about specific crops or fertilizers.
    All data are based on location and type of production.
    \item \textbf{Insert data}\\
    Insert data about their production, report every type of problems.
    \item \textbf{Request for help/suggestion}\\
    Farmers can request help with a text message that will be sent directly to the agronomists responsible of the area.
    \item \textbf{Create discussion forums}\\
    Create forums to discuss with the other farmers.
    In this section the creator can choose the name of the forum and invite all the desirable partecipants.
\end{enumerate}

\underline{Agronomists}
\begin{enumerate}
    \item \textbf{Insert area}\\
    Insert the area of responsibility for the agronomist.
    \item \textbf{Receive request for help/suggestion}\\
    Here the agronomist can manage all the incoming request for help or suggestion.
    This can be done with a specific section where the agronomist can visualize the message and answer it.
    \item \textbf{Visualize area stats}\\
    Visualize data about whether forecast or a list of best-performing farmers.
    The list of best-performing farmers is based on the productivity over a selected period of time.
    \item \textbf{Visualize and update daily plan}\\
    The daily plan consists in a list of farms to be visited during the day.
    Every farm must be visited at least twice a year with particular attention to the under-performing ones that 
    should be visited more often.
    \item \textbf{Confirm the daily plan}\\
    Confirm the daily plan at the end of the day or update it in case of deviations.
\end{enumerate}
\subsection{Definitions, acronyms, abbreviations}
\subsection{Revision history}
\subsection{Reference documents}
\begin{itemize}
    \item Specification document: "Assignment RDD AY 2021-2022"
    \item Alloy documentation: https://alloytools.org/documentation.html
    \item UML documentation: https://www.uml-diagrams.org/
    \item BPMN documentation: https://www.bpmn.org/
    \item Paper: "The World and the Machine" by M. Jackson and P. Zave
\end{itemize}
\subsection{Document structure}

\section{Overall Description}
\subsection{Product perspective}
\subsection{Product functions}
\subsection{User characteristics}
\subsection{Assumptions, dependencies and constraints}

\section{Specific Requirements}
\subsection{External Interface Requirements}
\subsubsection{User Interfaces}
\subsubsection{Hardware Interfaces}
\subsubsection{Software Interfaces}
\subsubsection{Communication Interfaces}

\subsection{Functional Requirements}
\subsection{Performance Requirements}
Test.

\subsection{Design Constraints}
Test.

\subsubsection{Standards compliance}
\subsubsection{Hardware limitations}
\subsubsection{Any other constraint}

\subsection{Software System Attributes}
\subsubsection{Reliability}
Test.

\subsubsection{Availability}
Test.

\subsubsection{Security}
\subsubsection{Maintainability}
\subsubsection{Portability}

\section{Formal Analysis using Alloy}
\section{Effort Spent}
    \begin{tabular}{| c || c | c| c| c |}
        \hline
        Student & Time for S.1 & S.2 & S.3 & S.4 \\ \hline
        Ottavia Belotti & 30min & ? & ? & ? \\
        Alessio Braccini & 2h & ? & ? & ? \\
        Riccardo Izzo & 2h & ? & ? & ? \\
        \hline
    \end{tabular}

\section{References}
\end{document}