\documentclass[table, 12pt]{article}
\usepackage[T1]{fontenc}
\usepackage[utf8]{inputenc}
\usepackage[english]{babel}
\usepackage{graphicx}
\usepackage{titlesec}
\usepackage{hyperref}
\usepackage[usenames,dvipsnames]{xcolor}
\usepackage{float}
\usepackage[export]{adjustbox}
\usepackage{longtable}

\hyphenation{Te-lan-ga-na}
\hyphenation{an-a-lys-ing}
\hyphenation{a-gron-o-mists}
\hyphenation{a-gron-o-mist}
\titleformat{\paragraph}
{\normalfont\normalsize\bfseries}{\theparagraph}{1em}{}
\titlespacing*{\paragraph}
{0pt}{3.25ex plus 1ex minus .2ex}{1.5ex plus .2ex}


\begin{document}
\begin{titlepage}
    \centering
    {\scshape\large AY 2021/2022 \par}
    \vfill
    \includegraphics[width=100pt]{assets/logo-polimi-new}\par\vspace{1cm}
    {\scshape\LARGE Politecnico di Milano \par}
    \vspace{1.5cm}
    {\huge\bfseries DD\@: Design Document \par}
    \vspace{2cm}
    {\Large {Ottavia Belotti\quad Alessio Braccini\quad Riccardo Izzo}\par}
    \vfill
    {\large Professor\par
        Elisabetta \textsc{Di Nitto}}
    \vfill
    {\large \textbf{Version 1.0}\\ \today \par}
\end{titlepage}

\hypersetup{%
    pdfborder = {0 0 0}
}

\thispagestyle{plain}
\pagenumbering{gobble}
\mbox{}
\newpage
\pagenumbering{roman}
\tableofcontents
\newpage
\pagenumbering{arabic}

\section{Introduction}


\subsection{Purpose}
The purpose of this document is to provide a full technical description of the system described in the RASD document.
In this design document we discuss about both hardware and software architectures in terms of interaction among the components that represent the system.
Moreover there are mentions about the implementation, testing and integration process.
This document will include technical language so is primarily addressed to programmers but stakeholders are also invited to read it in order to understand the characteristics of the project.

\subsection{Scope}
The scope of this design document is to define the behavior of the system in both general and critical cases, and to design the architecture of the system by describing
logical allocation of the components and the interaction between them.
This document also extends in part to the implementation and testing plan, where one possible course of action is explained, user interface design of user applications and requirements traceability relating to the RASD.
\subsection{Definitions, acronyms, abbreviations}
\subsubsection*{Acronyms}
\begin{itemize}
    \item \textbf{DREAM}: \emph{Data-driven predictive farming}
    \item \textbf{RASD}: Requirement Analysis and Specification Document
    \item \textbf{DD}: Design Document
    \item \textbf{API}: Application Programming Interface
    \item \textbf{DBMS}: Database Management System
    \item \textbf{UML}: Unified Modeling Language
    \item \textbf{GPS}: Global Positioning System
    \item \textbf{IT}: Information Technology
    \item \textbf{GUI}: Graphic User Interface
\end{itemize}
\subsection{Revision history}

\subsection{Reference documents}
\begin{itemize}
    \item Specification document: "Assignment RDD AY 2021-2022"
    \item Requirements Analysis Specification Document (RASD)
    \item UML documentation: https://www.uml-diagrams.org/
    \item ArchiMate documentation: https://pubs.opengroup.org/architecture/archimate3-doc/
    \item Slides of the lectures
\end{itemize}
\subsection{Document structure}
\begin{itemize}
    \item \textbf{Section 1} gives a brief description of the design document, it describes the purpose and the scope of it including all the definitions, acronyms and abbreviations used. 
    \item \textbf{Section 2} delves deeply into the system architecture by providing a detailed description of the components, the interfaces and all the technical choices made for the development of the application.
    It also includes detailed sequence, component and ArchiMate diagrams that describes in depth the system.
    \item \textbf{Section 3} contains a complete description of the user interface, it includes all the client-side mockups with some graphs useful to understand the correct execution flow.
    \item \textbf{Section 4} links the RASD and the DD, it maps the goals and the requirements described in the RASD to the actual functionalities presented in this DD.
    \item \textbf{Section 5} presents a description of the implementation, testing and integration phases of the system components.
\end{itemize}
\section{Architectural Design}


\subsection{Overview}
\begin{center}
    \begin{figure}[H]
        \includegraphics[scale=0.55, center]{assets/4-tier-scheme.jpg}
        \caption{Four-Tier Architecture Scheme}
        \label{fig: four-tier-architecture-scheme}
    \end{figure}
\end{center}

The system is a distributed application that follow the common client-server paradigm.
The architecture of the application is structured in three logic layers:
\begin{itemize}
    \item \textbf{Presentation Layer (P)}: it manages the presentation logic and handles the user actions.
    It is characterized by a GUI (Graphic User Interface) that allow the user to interact with the application in a simple and effective way.
    \item \textbf{Logic or Application Layer (A)}: it manages all the functionalities that has to be provided to the users, it is also responsible of data exchange between the client and the data sources.
    \item \textbf{Data Layer (D)}: it manages the access to data sources, it gets data from the database and move them through the other layers.
    It is essential to guarantee a high level of abstraction from the database in order to provide a model easy to use.
\end{itemize}

\begin{center}
    \begin{figure}[H]
        \includegraphics[scale=0.70, center]{assets/4-tier-architecture-scheme.jpg}
        \caption{High Level Architecture}
        \label{fig: high_level_architecture}
    \end{figure}
\end{center}

The system, as shown in \textit{Figure (\ref{fig: four-tier-architecture-scheme})}, is based on a four-tier architecture (Client, Web Server, Application Server and Database Server), this ensures more flexibility and high scalability.
The tiers are separated by firewalls in order to guarantee a higher level of security of the whole system.
A thin client is used to prevent heavy computation load client side, all the heavy operations are executed at the server side.
The client's devices can be a personal computer or a mobile device, both communicate directly with the Web Server through a web browser.
The Application Server communicates with the Database Server and transform data with business logic.
It also manages the news and the weather forecast services.
This allow to retrieve data like the weather directly from it in order to show them to the user.
Finally, to enable the geolocalization, the Application Server uses the API provided by Google Maps.
For what concern the other components, such as the water irrigation system and the humidity sensors, they are connected to the Database Server that exchange data with the Application Server.
The Web Server also manages the data coming from the irrigation system and the humidity sensors.
All the components will be described in depth in the following sections.

\subsection{Component view}
\subsubsection*{General Component View}
\begin{center}
    \begin{figure}[H]
        \includegraphics[scale=0.50, center]{assets/general-component-view.jpg}
        \caption{General Component Diagram}
        \label{fig: general_component_view}
    \end{figure}
\end{center}

This image gives an high level representation of the components of the system.
On the left are shown the provided interfaces between the Web Server and the Application Server that in this scheme is represented as a "black box", a complete description of it is provided in the next section.
All the interfaces basically represents the main functionalities requested by the client application.
On the right there are the requested interfaces, one of this is responsible of the geolocalization and is provided by the Google Maps API.
Moreover there are two interfaces that manage the weather forecast and the news services.
Finally, the DBMS interface manages the DBMS service and is responsible of the communication between the Application Server and the Database Server.

\subsubsection*{Application Server Component View}
\begin{center}
    \begin{figure}[H]
        \includegraphics[scale=0.45, center]{assets/application-server-component.png}
        \caption{Application Server Component Diagram}
        \label{fig: application_server_component_view}
    \end{figure}
\end{center}

The following component diagram gives a detailed view of the Application Server, it shows the internal structure and the interaction between the components.
External elements in the diagram are represented in a simplified way.
\begin{itemize}
    \item \textbf{AccountManager}: this component handles all the basic requests made by the client.
    This includes the authentication process that is composed by the log in, the log out and the sign up services. 
    It also includes the edit profile process responsible of the editing of the user account.
    Once a user is logged in, all the specific functionalities are provided by the component that manages this type of user.
    Finally it communicates with the \textit{LocationModule} that provides the geolocalization service.
    \item \textbf{LocationModule}: this module provides the interface that allow the geolocalization of the farm based on the GPS and on the address provided by the farmer.
    In order to provide this service it communicates with the map service manager.
    \item \textbf{MapServiceManager}: this component communicates directly with the external API provided by Google Maps, it provides informations regarding the region and allow the user to visualize the map of a specific location.
    Mainly it adapts the data received by the API in a comprehensible way for the other components, it also manages the API requests.
    \item \textbf{PolicyMakerManager}: it manages the policy maker services, these include: steering initiative, ranking, water irrigation and humidity sensors.
    The steering initiative service and the ranking service are mapped through an interface to the internal database, on the other side the water irrigation service and the humidity sensor service are considered external services and for this reason are linked to the external database.
    \item \textbf{AgronomistManager}: it manages the agronomist services, these include: answer help request, daily plan and ranking.
    It includes the weather external interface, it communicates with the weather component in order to provide the weather forecast service to the agronomist.
    The daily plan service and the ranking service are linked to the internal database, instead the answer request service communicates directly with the \textit{HelpRequestManager} component.
    \item \textbf{FarmerManager}: it manages the farmer services, these include: ask help request, production and forum.
    It includes two external interfaces: the first one communicates with the \textit{WeatherManager} component in order to provide the weather forecast service to the farmer, the other one communicates with the \textit{NewsManager} component.
    In this case only the production service is linked directly to the internal database, the ask request service is linked to the \textit{HelpRequestManager} component while the forum service expose an interface to connect to the \textit{ForumManager} component.
    \item \textbf{ForumManager}: this component manages the forum section, in particular it is responsible of the management of all the topics with the related messages between farmers.
    \item \textbf{DataManager}: it provides access to the external interface of the database, it manages queries and interacts with both the internal and the external databases.
    It includes two components: one is responsible of the operations with the external database, the other of the internal database.
    Both these components expose the related interface in order to use the service.
    This component is connected to the \textit{DBMSService} external component, this establishes the connection between the Application Server and the Database Server.
    \item \textbf{HelpRequestManager}: it manages the help requests between farmers and agronomists.
    It communicates with the \textit{FarmerManager} component for the help requests to ask, on the other way it communicates with the \textit{AgronomistManager} component for the ones to answer.
    Finally it provides an interface that communicate with the internal database in order to store the requests.
    \item \textbf{NewsManager}: this component manages the service related to the news, it provides updated suggestions about crops and fertilizers for the farmers.
    To do this it interacts with the external component \textit{NewsService}.
    \item \textbf{WeatherManager}: this component manages the service related to the weather forecast.
    It interacts with the external component \textit{WeatherForecastService} in order to get the weather data.
\end{itemize}

\subsection{Deployment view}

\subsection{Runtime view}

\subsection{Component interfaces}

\subsection{Selected architectural styles and patterns}

\subsection{Other design decisions}

\section{User Interface Design}


\section{Requirements Traceability}


\section{Implementation, Integration and Test Plan}


\section{Effort Spent}


\section{References}


\end{document}