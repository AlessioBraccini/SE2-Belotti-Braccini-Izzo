\documentclass[table, 12pt]{article}
\usepackage[T1]{fontenc}
\usepackage[utf8]{inputenc}
\usepackage[english]{babel}
\usepackage{graphicx}
\usepackage{titlesec}
\usepackage{hyperref}
\usepackage[usenames,dvipsnames]{xcolor}
\usepackage{float}
\usepackage[export]{adjustbox}
\usepackage{longtable}
\usepackage{mathtools}
\usepackage[most]{tcolorbox}
\usepackage{xparse}
\usepackage{subcaption}

\hyphenation{Te-lan-ga-na}
\hyphenation{an-a-lys-ing}
\hyphenation{a-gron-o-mists}
\hyphenation{a-gron-o-mist}
\hyphenation{da-ta-base}
\hyphenation{ca-u-sed}
\hyphenation{Sen-sor-Da-ta-Ma-na-ger}
\hyphenation{Ac-count-Ma-na-ger}
\hyphenation{Lo-ca-tion-Mo-dule}
\titleformat{\paragraph}
{\normalfont\normalsize\bfseries}{\theparagraph}{1em}{}
\titlespacing*{\paragraph}
{0pt}{3.25ex plus 1ex minus .2ex}{1.5ex plus .2ex}


\def\exampletext{Example}

\NewDocumentEnvironment{testexample}{ O{} }
{
\colorlet{colexam}{teal!60!black} % Global example color
\newtcolorbox[use counter=testexample]{testexamplebox}{%
    % Example Frame Start
    empty,% Empty previously set parameters
    title={\exampletext #1},% use \thetcbcounter to access the testexample counter text
    % Attaching a box requires an overlay
    attach boxed title to top left,
       % Ensures proper line breaking in longer titles
       minipage boxed title,
    % (boxed title style requires an overlay)
    boxed title style={empty,size=minimal,toprule=0pt,top=4pt,left=3mm,overlay={}},
    coltitle=colexam,fonttitle=\bfseries,
    before=\par\medskip\noindent,parbox=false,boxsep=0pt,left=3mm,right=0mm,top=2pt,breakable,pad at break=0mm,
       before upper=\csname @totalleftmargin\endcsname0pt, % Use instead of parbox=true. This ensures parskip is inherited by box.
    % Handles box when it exists on one page only
    overlay unbroken={\draw[colexam,line width=.5pt] ([xshift=-0pt]title.north west) -- ([xshift=-0pt]frame.south west); },
    % Handles multipage box: first page
    overlay first={\draw[colexam,line width=.5pt] ([xshift=-0pt]title.north west) -- ([xshift=-0pt]frame.south west); },
    % Handles multipage box: middle page
    overlay middle={\draw[colexam,line width=.5pt] ([xshift=-0pt]frame.north west) -- ([xshift=-0pt]frame.south west); },
    % Handles multipage box: last page
    overlay last={\draw[colexam,line width=.5pt] ([xshift=-0pt]frame.north west) -- ([xshift=-0pt]frame.south west); },%
    }
\begin{testexamplebox}}
{\end{testexamplebox}\endlist}


\begin{document}
\begin{titlepage}
    \centering
    {\scshape\large AY 2021/2022 \par}
    \vfill
    \includegraphics[width=100pt]{assets/logo-polimi-new.pdf}\par\vspace{1cm}
    {\scshape\LARGE Politecnico di Milano \par}
    \vspace{1.5cm}
    {\huge\bfseries Implementation Document \par}
    \vspace{2cm}
    {\Large {Ottavia Belotti\quad Alessio Braccini\quad Riccardo Izzo}\par}
    \vfill
    {\large Professor\par
        Elisabetta \textsc{Di Nitto}}
    \vfill
    {\large \textbf{Version 1.0}\\ \today \par}
\end{titlepage}

\hypersetup{%
    pdfborder = {0 0 0}
}

\thispagestyle{plain}
\pagenumbering{gobble}
\mbox{}
\newpage
\pagenumbering{roman}
\tableofcontents
\newpage
\pagenumbering{arabic}

\section{Introduction}
The code can be found in the official project repository on GitHub at the link: \url{https://github.com/AlessioBraccini/SE2-Belotti-Braccini-Izzo}.
\subsection{Purpose}

This document aims to describe how the implementation and integration
testing took place.
Implementation is the last step of the DREAM application
development cycle.
Testing, instead, means check that the critical parts of the application
works in a correct way, as described in the DD document.


\subsection{Definitions, Acronyms, Abbreviations}
\begin{itemize}
    \item API: Application Programming Interface
    \item DBMS: DataBase Management System
    \item DD: Design Document
    \item HTTP: HyperText Transfer Protocol
    \item JS: JavaScript
    \item REST: REpresentational State Transfer
    \item RASD: Requirements Analysis and Specification Document
    \item UI: User Interface
    \item URL: Uniform Resource Locator
    \item WSGI: Web Server Gateway Interface
    \item ASGI: Asynchronous Server Gateway Interface
\end{itemize}
\subsection{Revision History}
\begin{itemize}
    \item Version 1.0:
\end{itemize}
\subsection{References}
\begin{itemize}
    \item Django Framework: \url{https://www.djangoproject.com/}
    \item REST Framework: \url{https://www.django-rest-framework.org/}
    \item Vue.js: \url{https://vuejs.org/}
    \item Axios: \url{https://axios-http.com/docs/intro}
\end{itemize}


\section{Development}
\subsection{Implemented Functionalities}
Given the three types of user, we decided to implement th functionalities for Policy Maker users and Agronomist user. In particular:
\subsubsection*{Policy Maker}
\begin{itemize}
    \item Farmers ranking
    \item Visualization of humidity sensors and water irrigation systems data as graphs
    \item Retrieving agronomists' reports about steering initiatives
\end{itemize}
\subsubsection*{Agronomist}
\begin{itemize}
    \item Farmers ranking tailored on the agronomist's district
    \item Uploading reports concerning steering initiatives
    \item Creation and updating of daily plans
    \item Help requests inbox
    \item Weather widget
\end{itemize}
\subsection{Adopted Development Frameworks}
Model-View-Controller paradigm
\subsubsection{Programming Language}
The programming language of choice for the DREAM backend is Python.

For the client side we choose to use Javascript, a text-based programming language, that allows to build interactive web pages.
Alongside Javascript, that allow the interaction with the user of the elements present in the page, we used HTML and CSS to give structure and style to the page.

\begin{itemize}
    \item \textbf{Pros:}
    \begin{itemize}
        \item[+] Allow fast Development
        \item[+] Readability
        \item[+] Widely spread among WebApps  
    \end{itemize}
    \item \textbf{Cons:}
    \begin{itemize}
        \item[-] Slow 
    \end{itemize}
\end{itemize}

\subsubsection{Django Framework}

\subsubsection{Django REST Framework}

\subsubsection{Vue.js}

Vue.js is an open-source model–view–viewmodel front end JavaScript framework that allow to create user interfaces and single-page applications.
Vue.js features an incrementally adaptable architecture that focuses on declarative rendering and component composition.
The core library is focused on the view layer only.
Advanced features required for complex applications such as routing, state management and build tooling are offered via officially maintained supporting libraries and packages.

We used this framework to build up the client-side rendering of pages because its easy usage as it integrate in a single .vue file, also called components,
the HTML, CSS and javascript part.

In order to communicate with the backend we use the javascript library Axios.
It is a promise-based HTTP Client for node.js and the browser.
On the server-side it uses the native node.js http module, while on the client it uses XMLHttpRequests.
It transforms in an automatic way the json reply that arrives from the server in vue ready XMLHttpRequests.

\subsection{API Integration}
\subsubsection{OpenWeatherMap API}
To allow the user to retrieve the weather information we use this external api service that let us know in real time
the weather condition of a specific territory.
This return us not only the basics information but also more specific ones.

\subsection{DataBase}
Postgresql

\section{Source Code}
\subsection{Backend Structure}
\begin{center}
    \begin{figure}[H]
        \includegraphics[scale=0.65, center]{assets/backend_structure.png}
        \caption{Backend structure}
        \label{fig: backend_structure}
    \end{figure}
\end{center}

% basic description

\begin{itemize}
    \item \textbf{IT}: root directory, container for the project
    \item \textbf{dream\_backend}: folder that contains the configuration files of the project
    \item \textbf{\_\_init.py\_\_}: it tells the Python interpreter that the directory is a Python package
    \item \textbf{settings.py}: main setting file for the Django project, used to configure all the applications and middleware, it also handles the database settings
    \item \textbf{urls.py}: URL declarations for the Django project, it contains all the endpoints that the website should have
    \item \textbf{wsgi.py}: entry-point for WSGI-compatible web servers to serve your project, it describes the way how servers interact with the applications
    \item \textbf{asgi.py}: entry-point for ASGI-compatible web servers to serve your project, ASGI works similar to WSGI but comes with some additional functionality
    \item \textbf{migrations}: Django's way of propagating changes to the models into the database schema, when changes occur this folder is populated with the records of them
    \item \textbf{admin.py}: used for registering the Django models into the Django administration, it allows to display them in the Django admin panel
    \item \textbf{apps.py}: common configuration file for all Django apps, used to configure the attributes of the app
    \item \textbf{models.py}: it defines the structure of the database, it allows the user to create database tables for the app with proper relationships using Python classes. It tells about the actual design, relationships between the data sets and their attribute constraints
    \item \textbf{tests.py}: used to test the overall working of the app through unit tests
    \item \textbf{views.py}: provide an interface through which a user interacts with a Django website, it contains the business logic of the app
    \item \textbf{manage.py}: command-line utility for executing Django commands; these includes debugging, deploying and running
\end{itemize}

% describe apps
% flow of requests

\subsection{Frontend Structure}

The front-end web application is contained into the dream\_frontend folder of
the IT directory.
Of course, following the four tier architecture described in the Design Document,
the front-end web application can also be deployed to a dedicated web server,
which will then make requests to a different backend server.
Here is represented the structure of the web app:



\section{Testing}
\subsection{Backend Testing}

% \section{Build} %don't really have to build


\section{Installation}
\subsection{Requirements}
\subsection{Backend Installation}
\subsection{Frontend Installation}

\section{Effort Spent}
\begin{tabular}{|c||c|}
    \hline
    Student & Time for implementation\\ \hline
    Ottavia Belotti & 80h\\
    Alessio Braccini & 80h\\
    Riccardo Izzo & 80h\\
    \hline
\end{tabular}

\end{document}

